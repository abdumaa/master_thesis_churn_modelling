\documentclass[12pt,titlepage]{article}

\usepackage[english]{babel}
\usepackage[utf8]{inputenc}
\usepackage[T1]{fontenc}
\usepackage{color}
\usepackage[a4paper,lmargin={3cm},rmargin={2cm},tmargin={2cm},bmargin={2cm}]{geometry}
\usepackage[onehalfspacing]{setspace}
\usepackage{amssymb}
\usepackage{amsthm}
\usepackage[pdftex]{graphicx}
\usepackage{lmodern}
\usepackage{amsmath}
\usepackage{amsfonts}
\usepackage{mathrsfs}
\usepackage{graphicx}
\usepackage{tikz}
\usetikzlibrary{trees}
\graphicspath{{./images/}}


\begin{document}

\thispagestyle{empty}

\begin{titlepage}\centering
    \begin{center}
        \vspace*{\fill}
        \huge \textbf{\textsf{Customer Churn Prediction using Quotation Data}}\\
        \vspace{2cm}
        \LARGE\textbf{\textsc{Master Thesis}}\\
        \vspace{1cm}
        \normalsize
        Submitted on: \today \\
        \vspace{2.5cm}
        \large \textbf{at the University of Cologne}
        \vspace{3cm}
    \end{center}
    \normalsize{
        \begin{tabular}{ll}
            Name: & {Abdurahman Maarouf} \\
            Adress: & {Schulstrasse 31} \\
            Postcode, Area: & {53332, Bornheim} \\
            Country: & {Germany} \\
            Matriculation number: & {736481} \\
            Supervisor: & {Prof. Dr. Dominik Wied} \\
        \end{tabular}\\
    }
    \vspace*{\fill}

\end{titlepage}

\thispagestyle{empty}

\tableofcontents

\newpage

\pagenumbering{arabic}

\setcounter{page}{1}

\section{Introduction} \par

Predicting customer churn in order to retain customers has become one of the most important issues for companies.
The goal is to estimate probabilities for a costumer churning in the next period of time, in order to be able to detect
potential churners before they leave the company. To tackle this issue, more and more advanced Machine-Learning-Algorithms
are used guaranteeing high accuracy in their out-of-sample predictions. \\
Fortunately for most of the companies, churn rates from one period to another are very small. However in classification
models predicting a rare event can become challenging. In this so called "Imballanced Classes" issue certain arrangements
to the underlying training data need be made. Without these arrangements and with highly imballanced classes, a poor
algorithm will simply never predict the outcome of the minority class. In a dataset with 1000 customers containing 5
churners for example, this loss-minimizing algorithm would have an in-sample accuracy of 99.5\%. \\
In order to avoid the high amount of "False-Negative" classifications there are many methods ranging from upsampling the
minority class or downsampling the majority class to more advanced techniques. In this work we will present and compare
the different methods while applying them to the underlying problem. \\
We also want to emphasize (or not) the importance of using quotation data for predicting customer churn. A company can
track (potential) customer behavior on their distribution channels. Nowadays, in most cases the products or services are
offered online on websites, which makes it easy to track website visitor data. In the context of dealing with customer
churn this data can be matched to the customers already having a product or contract of this company. We believe (?) that
the number of visits of a current customer in the last period (?) plays a big role in predicting the probability of that
customer leaving in the next period. (Coming from high correlation between Nvisits and churn)\\
In order to evaluate the importance of not only the number of website visits but also the other explanatory variables there
is typically a trade-off during model selection. The trade-off is between the model complexity or corresponding accuracy
and the model interpretability. Deep neural networks or boosted trees belong to the complex models which are famous for
their high accuracy in the fields of computer vision and natural language processing. Understanding and interpreting the
model is of no big interest in these areas. However in the topic of this work and in many other areas understanding which
variables lead to the resulting outcome of the model becomes desirable. The most transparent models in terms of
interpretability are linear or logistic models. There the magnitude and sign of the corresponding coefficients (after being
testet for significance) illustrate the changes of the outcome for a change in the specific explanatory variable. These models
however lack in terms of accuracy when being compared to the comlex ones. In this work we will present the accuracy and
interpretability of "Explainable Boosting Machines" developped by (?) for predicting customer churn. It aims to combine the
high accuracy of complex models on the one hand and the interpretability of linear models on the other hand. \\

\section{Data and Methodology} \par

\subsection{Understanding the Problem} \par

For this work we use customer data from a big insurance company in Germany. Due to data pretection the data is anonymized which
does not affect the model accuracy and interpretability in any form. We focus on the product of automobile liability insurance,
which is by law a mandatory service every car owner must hold in Germany. \\
Typically car owners close a deal with an insurance company which can be terminated by the end of each year. In rare cases both
sides agree on a contract with a due date during the year. If the contract does not get terminated it is automatically extended
for another year. Besides the option to terminate the contract at the due date there is also an option to terminate
it earlier in a few special cases. These cases mainly involve car accidents and vehicle changes of the contractor. To sum up,
here are the three cases in which a churn can (but not must) occur during a year: \\

\begin{center}
    \begin{tabular}{ll}
        Event A: & Contractor is involved in an Accident. \\
        Event N: & Contractor buys a new Car. \\
        Event D: & Due date during the year. \\
    \end{tabular}
\end{center}

The problem of modelling customer churn needs to be seperated into the probability of a costumer leaving during the
year and at the end of a year (why noch ausführen). In this work we will focus on predicting churns occuring during the year.
The purpose is to build a model which can be used at any time $t$ of the year besides on January the 1st to predict the probability of
a costumer leaving the company in the next period of time $(t, t+s]$. \\
It can be argued that in order to provide a model with maximized utility for production one would want to keep $s$ small. For example
a company would highly benifit from a model, which can predict the churn-probability of tomorrow or the next week. However we will see that
having a small $s$ will decrease the accuracy of our models (drastically?), creating a trade-off situation between model accuracy and the benifits
of a small $s$. With a smaller period the classes of the data become more imballanced, creating a higher challange of preprocessing the training
data and feeding the algorithm enough information on potential churners. Furthermore, a small $s$ decreases the scope of action for a company to
retain potential customers leaving. \\

Figure 1 (richtiger Verweis) illustrates how the probability of a churn during the year can be decomposed using the Events A (Accident), N
(New Car), D (Due date during the year) and C (Churn). \\

% Set the overall layout of the tree
\tikzstyle{level 1}=[level distance=5cm, sibling distance=3cm]
\tikzstyle{level 2}=[level distance=5cm, sibling distance=2cm]

% Define styles for bags and leafs
\tikzstyle{bag} = [text width=4em, text centered]
\tikzstyle{end} = [circle, minimum width=3pt,fill, inner sep=0pt]

% The sloped option gives rotated edge labels.
\begin{tikzpicture}[grow=right, sloped]
\node[bag] {}
    child {
        node[bag] {$None$}
            child {
                node[end, label=right:
                    {$P(None\cap \overline{Churn})$}] {}
                edge from parent
                node[above] {$P(\overline{C}\mid None)$}
            }
            child {
                node[end, label=right:
                    {$P(None\cap Churn)$}] {}
                edge from parent
                node[above] {$P(C\mid None)$}
            }
            edge from parent
            node[above] {$P(None)$}
    }
    child {
        node[bag] {$Due$ $Date$}
        child {
                node[end, label=right:
                    {$P(Due$ $Date\cap \overline{Churn})$}] {}
                edge from parent
                node[above] {$P(\overline{C}\mid D)$}
            }
            child {
                node[end, label=right:
                    {$P(Due$ $Date\cap Churn)$}] {}
                edge from parent
                node[above] {$P(C\mid D)$}
            }
        edge from parent
            node[above] {$P(D)$}
    }
    child {
        node[bag] {$New$ $Car$}
        child {
                node[end, label=right:
                    {$P(New$ $Car\cap \overline{Churn})$}] {}
                edge from parent
                node[above] {$P(\overline{C}\mid N)$}
            }
            child {
                node[end, label=right:
                    {$P(New$ $Car\cap Churn)$}] {}
                edge from parent
                node[above] {$P(C\mid N)$}
            }
        edge from parent
            node[above] {$P(N)$}
    }
    child {
        node[bag] {$Accident$}
        child {
                node[end, label=right:
                    {$P(Accident\cap \overline{Churn})$}] {}
                edge from parent
                node[above] {$P(\overline{C}\mid A)$}
            }
            child {
                node[end, label=right:
                    {$P(Accident\cap Churn)$}] {}
                edge from parent
                node[above] {$P(C\mid A)$}
            }
        edge from parent
            node[above] {$P(A)$}
    };
\end{tikzpicture} \par

By assumption we set $P(C\mid None) = 0$ as the amount of terminated contracts during the year which are not being
caused by a new car, an accident or a due date is very small and can be omitted. Therefore we leave these cases out of our data
(?). Also, the probability $P(D)$ can only take values $0$ and $1$, as either the due date of a customer lies in the next period
of time $(t, t+s]$ or not. What we are interested in predicting is the overall probability of a churn, which can be rewritten as:

\begin{center}
    \begin{tabular}{ll}
        $P(C)$ & $=$ \hspace{3mm} $P(A\cap C) + P(N\cap C) + P(D\cap C)$ \\
        & $=$ \hspace{3mm} $P(A)P(C\mid A) + P(N)P(C\mid N) + P(D)P(C\mid D)$ \\
    \end{tabular}
\end{center} \par

One idea would be to model the three branches of $A$, $N$ and $D$ seperately. The logic behind this is that different models and
sets of explanatory variables have the best fit for the probabilities of the three branches. Not only the the three branches
but also the unconditional and conditional probabilities of a single branch may vary in their best modelling approach. For prediciting
an accident a model of type (XY) is more suitable, whereas (YZ) would go better with modelling the probability of churn given an
accident occured. In the course of this work we will begin with a general modelling approach trying to predict the probability $P(C)$.
At a later stage we will compare the accuracy outcomes of seperately predicting the branch probabilities with the baseline approach. \\

\subsection{Data} \par

Most customer churn predicition models have been using only one timestamp to train a model. We want to emphasize the usage of multiple timestamps and
show that it significantly improves prediction accuracy. The improvement is a result of an increased training set size and the ability of the model to
especially learn more about the rare class. Furthermore, the model becomes more generalizable in time which is crucial if this model is applied in
production. (see: How training on multiple time slices improves performance in churn prediction). Explain more. Use Graphics like in the paper.\\
Part of this work will also be to evaluate if the churn probability of customers is time invariant. Therefore we statistically test the equality of
monthly and yearly (and weekly?) mean churn rates using the ANOVA (or other? seasonality tests). We will see that it is time variant (or not?) which underlines the
importance of including (or not) different timestamps containing different years/months(/weeks) in the data for the model to learn the time differences
(or not). (These will simply be appended to the data as additional rows creating a panel dataset.) \\
Therefore to build the models we use historical data of the insurance company. More specifically we pick $N$ (one or many?) timestamp(s) $t_{i}$ with
$i = 1,...,N$ in the past and collect all the active contracts to that(these) timestamp(s). One row corresponds to one active contract at $t_{i}$. So
if one hypothetical contract is active in all $N$ periods it will appear as $N$ seperate rows in our data. To each row we merge the status of that
contract in $t_{i}+s$. Furthermore, we join the number of requests corresponding to that contract in the period $[t_{i}-m, t_{i}]$. The period length $m$
will be another parameter to tune. \\
Describe the ETL Process? \\

-Statt "STORNO" im data-load sql query zu definieren, definieren wir es einfach in dem Feature-Engineering Teil?\\
-Test if customer churn probability is time invariant or time variant!!! Is it enough to argue with the Aggregate Churn rate? Show statistically (Voll gute Idee)\\
-How do I exclude 1.1. churners in my data? Exclude the entire contract or simply say y=0 as he didnt churn during the year?\\
-Safe noch als Variable einbauen, was sein Preis bei einer jetzigen Berechnung wäre (und dann Differenz zwischen Preis, den er zahlt und Preis den er angezeigt bekommt!)\\

\section{Modelling Approach} \par

\subsection{Preprocessing} \par

\subsection{Handling Class Imbalance} \par

Studying the rarity of an event in the context of machine learning has become an important challange in the recent two decades. Rare events, such as a customer churning in the
next period, are much harder to identify and learn for most of the models. HaiYing Wang et al (Logistic Regression for Massive Data with Rare Events) study the convergence rate
and distributaional properties of a Maximum-Likelihood estimator for the parameters of a logistic regression while prediciting rare events. Their finding is that the convergence
rate of the MLE is equal to the inverse of the number of examples in the minority class rather then the overall size of the training data set. So the accuracy of the estimates for
the parameters is limited to the available information on the minority class, even if the size of the dataset is massive. \\
Therefore some methods have been developped to decrease the problematic of imballanced classes. In
this chapter we will present (three?) different methods which can be applied to the training set, before feeding it to the model. To handle and evaluate the outcomes
of prediciting rare events also the appropriate models and model evaluation metrics must be chosen. This will be discussed in the next two chapters. \\

\subsubsection*{(i) Downsampling}

The first basic sampling method is named downsampling. It randomly eliminates examples from the majority class in order to artifficially decrease the imballance between the
two classes. The downside of this approach is that it possibly eliminates useful examples for the model to maintain a high accuracy in predicting the majority class 
(Mining with Rarity: A Unifying Framework). HaiYing Wang et al (Logistic Regression for Massive Data with Rare Events) also study the convergence rate and distributaional
properties when applying downsampling. According to their findings the asymptotic distribution of the resulting parameters may be identical to the MLE's using the full data set.
Under this condition there is no loss in terms of efficiency (minimum possible variance of an unbiased estimator devided by its actual variance). \\

\subsubsection*{(ii) Upsampling}

The second basic sampling method is the upsampling approach. This method simply duplicates examples from the minority class until the classes are more balanced.
While duplicating examples though, the chances of overfitting to these duplicates becomes a more probable threat.
Also, no new data is being generated in order to let the model learn more valuable characteristics about the minority class (Mining with Rarity: A Unifying Framework).
Additionally, the computational performance of this approach can get rather poor, espacially with large datasets and highly imballanced classes. While evaluating the
asymptotics of the MLE's with upsampling, HaiYing Wang et al find out that it also decreases the efficiency. A probable higher asymptotic variance of the estimators
is the reason for that. \\


\subsubsection*{(iii) SMOTE}

\subsection{Machine Learning Models} \par

\subsection{Model Evaluation Metrics} \par

\newpage

\thispagestyle{empty}

\section*{Bibliography}
\vspace*{6mm}

\end{document}s